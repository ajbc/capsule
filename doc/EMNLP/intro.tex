%!TEX root = emnlp2016.tex

Foreign embassies of the United States government communicate with one
another and with the U.S. State Department through diplomatic cables.
The National Archive collects these cables in a corpus, which traces
the (unclassified) diplomatic history of the United States. The corpus
contains, for example, over two million cables sent between 1973 and
1978.

Most of these cables describe diplomatic ``business as usual,'' such
as arrangements for visiting officials, recovery of lost or stolen
passports, or obtaining lists of names for meetings and
conferences. For example, the embassies sent 8,635 cables during the
week of April 21, 1975. Here is one, selected at random:
\begin{shaded*} \tt{Hoffman, UNESCO Secretariat, requested info from
PermDel concerning an official invitation from the USG
RE subject meeting scheduled 10--13 JUNE 1975, Madison,
Wisconsin.  Would appreciate info RE status of action to
be taken in order to inform Secretariat.  Hoffman communicating
with Dr.~John P.~Klus RE list of persons to be invited.}
\end{shaded*}

But, hidden in the corpus are also cables about important diplomatic
events---the cables and events that are most interesting to
historians, political sceintists, and journalists. For example, during
that same week, the U.S. was in the last moments of the Vietnam war
and, on April 30, 1975, lost its hold on Saigon. This triggered the
end of the war and a max exodus of refugees. Here is one of the cables
about this event:
\begin{shaded*}
  \tt{GOA program to move Vietnamese Refugees to Australia
  is making little progress and probably will not cover more than
  100-200 persons.  Press comment on smallness of program has
  recognized difficulty of getting Vietnamese out of Saigon, but
  ``Canberra Times'' Apr 25 sharply critical of government's
  performance.  [...]
  %Opposition continues to attack smallness of program,
  %but seems concerned, as does government, with scoring
  %point of domestic political importance.  With Parliament in
  %recess for next three weeks and Prime Minister on trip, issue
  %may attract less attention.
  Labor government clearly hopes whole
  matter will somehow disappear.}
\end{shaded*}

% cheating by putting this here...
\begin{figure*}[ht]
\centering
\includegraphics[width=\linewidth]{fig/cables_events.pdf}
\caption{Measure of ``eventness,'' or time interval impact on cable content (Eq.~\ref{eq:eventness}).  Grey background indicates the number of cables sent over time.  This comes from the model fit we discuss in \Cref{sec:eval}.  Capsule successful detects real-world events from National Archive diplomatic cables.}
\label{fig:cables_events}
\end{figure*}

Our goal in this paper is to develop a tool to help historians,
political scientists, and journalists wade through corpora of
documents to find potentially significant events and the primary
sources around them. We present \textit{Capsule}, a probabilistic
model for detecting and characterizing important events, such as the
fall of Saigon, in large corpora of historical communication, such as
diplomatic cables from the 1970s.

\Cref{fig:cables_events} illustrates Capsule's analysis of two million
cables from the National Archives' corpus. The \mbox{$y$-axis}
represents ``eventness,'' a loose measure of how strongly a week's
cables deviate from typical diplomatic ``business as usual'' to
discuss some matter that is common to many embassies. (We describe
this measure of ``eventness'' in detail in section~\ref{sec:model}.)

% THIS IS UNTOUCHED...
The figure shows that Capsule detects many of the important moments
during this five-year span, including the Air France hijacking and
Israeli rescue operation ``Operation Entebbe'' (June 27--July 4,
1976), and the fall of Saigon (April 30, 1975). It also identifies
other moments, such as the U.S. sharing lunar rocks with other
countries (March 21, 1973) and the death of Mao Tse-tung (Sept. 9,
1976). Broadly speaking, Capsule gives a picture of the diplomatic
history of these five years; it identifies and characterizes moments
and source material that might be of interest to a historian.

The intuition behind Capsule is this: Embassies write cables
throughout the year, usually describing typical diplomatic business,
such as visits from government officials. Sometimes, however,
important events occur, such as the fall of Saigon, that pull
embassies away from their typical activities and lead them to write
cables that discuss these events and their consequences. Capsule
therefore operationalizes an ``event'' as a moment in history when
multiple embassies deviate from their usual topics of discussion and
each embassy deviates in the same way.

Capsule embeds this intuition into a Bayesian model that uses latent
variables to encode what ``business as usual'' means for each embassy,
to characterize the events of each week, and to identify the cables
that discuss those events. Given a corpus of cables, the corresponding
posterior distribution of the latent variables provides a filter for
the cables that isolates important moments in diplomatic
history. \Cref{fig:cables_events} depicts the mean of this posterior
distribution.

We present the Capsule model in section~\ref{sec:model}, providing
both a formal model specification and guidance on how to use the model
to detect and characterize real-world events. In
section~\ref{sec:valid}, we validate Capsule using simulated data, and
in section~\ref{sec:eval}, we use it to analyze over two million
U.S. State Department cables. Although we describe Capsule in the
context of diplomatic cables, it is suitable for exploring any corpus
with the same underlying structure: text (or other discrete
multivariate data) generated over time by known entities. This
includes email, consumer behavior, social media posts, and opinion
articles.

\section{Related Work}

% I prefer having this as a stand-alone section

%\parhead{Related work.}
We first review previous work on automatic
event detection and other related concepts.

% While Capsule uses text documents and associated metadata as input, event detection is often performed with univariate input data.  In this context, bursts that deviate from typical behavior (e.g., noisy constant or a repeating pattern) can define an event \cite{kleinberg2003bursty,ihler2007learning}; Poisson Processes~\cite{Kingman:1993} are often used to model events under this definition.  Alternatively, events can be construed as ``change points'' to mark when typical observations shift semi-permanently from one value to another~\cite{guralnik1999event}.
In both univariate and multivariate settings, the goal is often that analysts want to predict whether or not rare events will occur~\cite{weiss1998learning,das2008anomaly}.  Capsule, in contrast, is designed to help analysts explore and understand the original data: our goal is interpretability, not prediction.

Events can also be construed as ``change points'' to mark when typical observations shift semi-permanently from one value to another~\cite{guralnik1999event,adams2007bayesian}. Both varieties of events are important, but we focus on temporary shifts away from normal.

% Text is often used in event detection, as it is an abundant source of data.
% In some applications, documents themselves are considered to be observed events~\cite{mccallum1998comparison,peng2007event}, or events are predetermined and tracked through the documents~\cite{yang2000improving,VanDam:2012}.  We are interested in detecting \emph{unobserved} events which can be characterized by patterns in the data.
%\newpage % note:when using hyperref, references can be split between pages!
A common goal is to identify clusters of documents; these approaches are used on news articles~\cite{zhao2012novel,zhao2007temporal,zhang2002novelty,li2005probabilistic,wang2007mining,allan1998line} and social media posts~\cite{VanDam:2012,lau2012line,jackoway2011identification,sakaki2010earthquake,reuter2012event,becker2010learning,sayyadi2009event}.
In the case of news articles, the task is to create new clusters as novel news stories appear---this does not help disentangle typical content from rare events of interest.
Social media approaches identify rare events, but the methods are designed for short, noisy documents; they are not appropriate for larger documents that contain information about a variety of subjects.

Many existing methods use document terms as features, usually weighted by tf-idf value~\cite{fung2005parameter,kumaran2004text,brants2003system,das2011dynamic,zhao2007temporal,zhao2012novel}; here, events are bursts in groups of terms. % Because language is high dimensional, using terms as features limits scalability.

Topic models~\cite{Blei:2012} reduce the dimensionality of text data; they have been used to help detect events mentioned in social media posts~\cite{lau2012line,dou2012leadline} and posts relevant to monitored events~\cite{VanDam:2012}.
We rely on topic models to characterize both typical content and events, but grouped observations can also be summarized directly~\cite{peng2007event,chakrabarti2011event,gao2012joint}.

In addition to text data over time, author~\cite{zhao2007temporal}, news outlet~\cite{wang2007mining}, and spatial information~\cite{Neill:2005,mathioudakis2010identifying,liu2011using} can be used to augment event detection.  Capsule uses author information in order to characterize the typical concerns of authors.

Detecting and characterizing relationships~\cite{schein2015bayesian,linderman2014discovering,das2011dynamic} is related to event detection.  When a message recipient is known, Capsule can use a sender-receiver pair in place of an author, but the model could be further tailored for network interactions.

% TODO: cite he2015hawkestopic
